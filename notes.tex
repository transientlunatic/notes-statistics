\documentclass{dwnotes}         		        % The Tufte-book class
\usepackage{danielphysics}
\usepackage{algpseudocode}
\title{Statisitical Astronomy}
\author{Daniel Williams}
\pubyear{2013}
\date{}
\begin{document}
\maketitle

\tableofcontents

\section{Introduction}
\label{sec:intro}

This course investigates how to make general statements about the
universe based on specific examples of its behaviour, which is common
to all sciences. \\ Astronomy and astrophysics are not generally an
experimental science -- relying almost entirely on remote sensing to
collect data, and so most work goes ahead by plausible reasoning based
on the available information, rather than through a more frequentist
and repeatious approach.

\section{Plausible Reasoning}
\label{sec:reasoning}

\subsection{Types of Reasoning}
\label{sec:types}

Consider the following three statements:
\begin{enumerate}
\item All stars in the Pleiades are within 200 parsecs of the
  Earth. (Hypothesis $C$)
\item The star Alcyone is in the Pleiades. ($B$)
\item Alcyone is within 200 pc of the Earth. ($A$)
\end{enumerate}
There is clearly some sort of relationship between these
statements. Let us assume that statement 1 is true, which allows us to
define a {\em hypothesis space}, then,
\begin{itemize}
\item If $B$ is true, this implies that $A$ is true.
\item If $A$ is flase, this implies that $B$ is false.
\end{itemize}
But if $A$ and $C$ are true, what can we infer about $B$? Nothing
definite, but,
\begin{itemize}
\item If $A$ is true, then $B$ is {\em more plausible}.
\item If $B$ is false, then $A$ is less plausible.
\end{itemize}

\subsection{Boolean Algebra}
\label{sec:boolean}

Both deductive and plausible reasoning follow the same algebraic rules
of logic which were set down in 1854 by George Boole. We can denote a
set of statements as $A, B, C, \dots$, and define a number of
operations.
\begin{itemize}
\item \textbf{logical product} $AB=$``both $A$ and $B$ are true.''
\item \textbf{logical sum} $A+B=$``at least one of $A$ or $B$ are
  true.''
\item \textbf{logical negation} $\bar{A}=$ ``A is false.''
\end{itemize}
This allows the definition of a number of theorems,
\begin{enumerate}
\item $A(B+C) = AB + AC$
\item $A + AB = A$
\item $A + BC = (A+B)(A+C)$
\item $A + \bar A = \text{True}$
\item $A \bar A = \text{False}$
\item $\bar{\bar{A}} = A$
\item $A + \bar A \bar B = A \bar B$
\item $(\overline{A+B}) = \bar A \bar B$
\item $(\overline{AB}) = \bar A + \bar B$
\end{enumerate}
We can go on to add a further operator to account for plausible
reasoning; the conditional operator:
\begin{itemize}
\item \textbf{logical condition} $A|B=$ ``the conditional probability
  that $A$ is true given that $B$ is true.''
\end{itemize}
In the Pleiades example of section \ref{sec:reasoning},
\begin{equation}
  \label{eq:pleiades1}
  B | CA > B | C
\end{equation}
\begin{equation}
  \label{eq:pleiades2}
  A | \bar{B} C < A | C
\end{equation}
These degrees of plausibility can be mapped onto a continuous and
monotonically increasing real function in the range $p(x) \in [0,1]$.
Thus,
\begin{equation}
  \label{eq:plausibilitysum}
  p(A|B) = p(\bar A | B) = 1
\end{equation}
\begin{equation}
  \label{eq:plausibilityproduct}
  p(AB|C) = p(A|C)p(B|AC) = p(B|C)p(A|BC)
\end{equation}
\begin{figure}
  \centering
  \definecolor{accent-pink}{HTML}{C3325F}
\definecolor{accent-red}{HTML}{9d261d}
\definecolor{accent-purple}{HTML}{7a43b6}
\definecolor{muted-blue}{HTML}{195073}
\definecolor{muted-green}{HTML}{7F8C1F}
\definecolor{muted-orange}{HTML}{EE913F}

\begin{tikzpicture}[scale=1.0]

\fill [accent-pink] (0,0) ellipse (4 and 2);
\draw [white] (-2, 1) node {C};
\fill [muted-blue, opacity=0.6] (1,.5) ellipse (2 and 1);
\draw [white] (1,.5) node {A};
\fill [muted-orange, opacity=0.6] (-1,-.5) ellipse (2 and 1);
\draw [white] (-1,-.5) node {B};

\end{tikzpicture}

  \caption{A Venn diagram illustrating the conditional operations.}
  \label{fig:conditionalvenn}
\end{figure}
which can be expressed in terms of a Venn diagram (viz figure
\ref{fig:conditionalvenn}).  If $A$ and $C$ are true, only a fraction,
$\frac{AB|C}{A|C}$ of this certainty correspnds to $B$ also being
true, i.e. $\frac{p(AB|C)}{p(A|C)} = p(B|AC)$.

\subsection{Bayes' Theorem}
\label{sec:bayestheorem}

It is thus possible to re-express the product rule as
\begin{equation}
  \label{eq:bayestheorem}
  p(B|AC) = \frac{ p(B|C) p(A|BC) }{ p(A|C) }
\end{equation}
This is Bayes' Theorem (Rev Thomas Bayes, 1763) and is used to assess hypotheses in the light of new evidence. Historically $p$ has been known as the ``probability'', and the process which uses it is known as Bayesian probability theory. \\
This interpretation of probability is simultaneously the oldest and
the youngest: it was put forward by Laplace, Bernoulli, and Bayes, but
then fell out of favour, until the period between 1939 and 1950 when
it was put onto a firmer, more modern mathematical footing. In the
intervening period frequentist probability came to be the more
commonly used interpretation.

\begin{example}
  Harold may have a rare disease, present in 1 out of 10,000 people of
  his age and background. He has a blood test which is known to show a
  positive result in 95\% of cases when the person has the disease,
  but on 1\% of cases when they do not.
  Harold recieves a positive result; what is the probability that he has the disease?

  Let $H = \text{Healthy}$, $D = \text{Diseased}$, $+ = \text{positive result}$, $- = \text{negative result}$. \\
  Let $C = \text{Harold is a typical member of the population}$, then,
  \begin{align*}
    p(D|C) &= 0.0001 \\
p(+|DC) &= 0.95 \\
p(+|HC) &= 0.01 \\
p(H|C) &= 0.9999
  \end{align*}
We want to know $p(D|+C)$.
By Bayes' Theorem, equation (\ref{eq:bayestheorem}),
\[ p(D|+C) = p(D|C) \frac{p(+|DC)}{p(+|C)} \] we know everything on
the right hand side except $p(+|C)$, the probability of a positive
result for a group member. Each group member is either ill, or not
ill, so
\[ p(+|C) = \overbrace{p(H|C) p(+|HC)}^{\text{Healthy}} 
   + \overbrace{p(D|C) p(+|DC)}^{\text{Diseased}} \]
so,
\[ p(D|+c) = \frac{0.0001 \times 0.95}{0.9999 \times 0.01 + 0.0001 \times 0.95} = 0.01\]
So he has a 1\% chance of being diseased. The moral o this story is \[ p(+|DC) \neq p(D|+C) \]
\end{example}

\begin{example}
  What is $p(A+B|C)$ in terms of $p(A|C) + p(B|C)$?  We can evaluate
  this by multiple applications of the product and sum rules.
  \\Recalling $A+B = \overline{\bar{A} \bar{B}}$, and using $p(X|Y) +
  p(\bar{X}|Y) = 1$,
\[ p(A+B | C) + p(\bar{A} \bar{B} | C) = 1 \]
so
\begin{align*}
   p(A+B|C) &= 1 - p(\bar{A}\bar{B}| C)\\
&= 1 - p(\bar{A}|C) p(\bar{B}| \bar{A}C)\\
&= 1 - p(\bar{A}| C) \left[ 1 - p(B|bar{A}C) \right] \\
&= p(A|C) + p(B|C) p(\bar{A}| BC) \\
&= p(A|C)+p(B|C)[1-p(A|BC)]
\end{align*}
thus
\[ p(A+B|C) = p(A|C)+p(B|C)-p(AB|C) \]
which is the extended sum rule.
\end{example}

\section{Frequentist Probability Theory}
\label{sec:frequentistcrap}

Bayesian probability is overtly subjective. It works with states of
belief, so different people will assert different probabilities to the
same statement if their state of knowledge is different. Frequentist
probability is an attempt to remove this subjectivity.

\subsection{Frequentist Probability}
\label{sec:freqprob}

Suppose we perform an experiment $N$ times (e.g. tossing a coin), and we define the relative frequency of an outcome with attribute $A_i$, (e.g. `heads'), as 
\[ f(A_i) = \frac{n(A_i)}{N} \]
Then, the probability of an outcome $A_i$ is
\begin{equation}
  \label{eq:probfreq}
  p_f(A_i) = \lim_{N \to \infty} \frac{n(A_i)}{N}
\end{equation}
We need to make a few assumptions:
\begin{itemize}
\item All experiments are conducted under the same conditions.
\item The limit converges.
\item Past frequencies predict future frequencies.
\end{itemize}

\subsection{Combinatorial Probability}
\label{sec:combinprob}

Suppose we toss a dice which we believe to be fair, so the combinatorial probability of tossing a 6 is
\[ P_{\rm c} = \frac{\text{Number of ways to toss 6}}{\text{Number of
    possible outcomes.}} = \frac{1}{6} \] However, this is still a
circular definition (`fair' means equally probable), and we need to be
able to enumerate the total number of outcomes. As $N \to \infty$ there is a good probability that the frequentist probability will converge on the combinatorial.\\
Both $p_{\rm f}$ and $p_{\rm c}$ obey the sum and product rules, and
both are a valid starting point for assigning `degrees of belief' in
Bayesian probability theory.

\begin{example}
  There are two urns, each containing red and white balls.
  \begin{itemize}
  \item Urn 1 contains 3 red and 7 white balls
  \item Urn 2 contains 6 red and 4 white balls
  \end{itemize}
  You, blindfolded, extract a ball from an urn. What is the probability
  \begin{itemize}
  \item that it is red?
  \item that it came from urn 1, given it was red?
  \end{itemize}

  Using Bayes' theorem, but assigning the probabilities using
  combinatorics,
  \[ p(R|1) = \frac{3}{10}, \quad p(R|2) = \frac{6}{10} \quad p(1) =
  p(2) = \frac{1}{2} \]
  Then,
  \[ p(R) = p(R|1)p(1) + p(R|2)p(2) = \frac{3}{10} \cdot \frac{1}{2} +
  \frac{6}{10} \cdot \frac{1}{2} = 0.45 \] Thus, by Bayes,
  \[ p(1|R) = \frac{p(1)p(R|1)}{p(R)} = \frac{0.5 \times 0.3}{0.45} =
  0.3 \]
\end{example}

\section{Probability Distributions}
\label{sec:distributions}

Thus far we have looked at situations that are true or false; with the
probability distributed between two options (BPT) or with a frquency
of heads/tails (FPT). What if there are any outcomes to consider? (eg
the number of photons arriving in a time $\tau$ from a source). To do this we distribute the probability arounf the options, defining
\[ p(0), p(1), \dots, p(\infty) \] with \[ \sum_{i=0}^{\infty} p(i) =
1 \] In FPT an observed event with several possible outcoes is called
a random event. If the outcome is a measurable quantity it is a random
variable. In BPT randomness just implies a lack of knowledge.

If a random variable can take only a finite number of states it is a discrete random variable. We can associate a probability, $p(r)$, with each of the outcomes, $r$. The set of all $p(r)$ is the probability distribution of the discrete random variable, $r$.

\subsection{Poisson Distribution}
\label{sec:poisson}

A Poisson random variable obeys the following axioms:
\begin{enumerate}
\item The probability of an event occuring in a time interval $\tau$ is independent of any past event.
\item Events occur at an intrinsic rate, $\mu$, such that the probability of a single event in the time $\delta t$ is $\mu \delta t$.
\item The probability of two (or more) events occuring at the same time is $0$.
\end{enumerate}
\begin{equation}
  \label{eq:poisson}
  p(N|t, \mu) = \frac{(\mu\tau)^N}{N!} \exp(-\mu \tau)
\end{equation}
\begin{figure}
  \centering
  \begin{tikzpicture}
    \begin{axis} [
    ybar, 
    width=\linewidth, 
    height=2in,
    domain=-2:14,
    samples=16,
    xmin=-2, xmax=14
    ]
    \addplot gnuplot[id=poi, mark=none, samples=16, accent-blue, ultra thick]{
    	     exp(x*log(5)-lgamma(x+1)-5)
    };
    \end{axis}
\end{tikzpicture}
  \caption{The Poisson distribution.}
  \label{fig:poisson}
\end{figure}
{\em see proof on handout.}\\
\begin{example}
  On an image of the sky stars are distributed randomly with one star,
  on average, per solid angle, $\Omega$.  Drawing a circle on the
  image with solid angle $6\Omega$, what is the probability that
  \begin{enumerate}
  \item it contains exactly 6 stars?
  \item it contains 10 or fewer stars?
  \end{enumerate}

  Using the Poisson distribution with $\mu = \frac{1}{\Omega}$ and
  $\tau = 6\Omega{}$, 
\begin{align*} p(N) &= \frac{\left( \frac{1}{\Omega} \cdot 6\Omega\right)^{N} \exp\left( -\frac{1}{\Omega}\cdot 6\Omega \right)}{N!} \\ &= \frac{6^N}{N!} \exp(-6) \\ p(6) &= \frac{6^6}{6!} \exp(-6) = 0.16
\end{align*}
Then, 
\begin{align*}p(N \le 10) &= p(0) + p(1) + \dots + p(10) \\ &= 0.89
\end{align*}
\end{example}

\subsection{Uniform Distribution}
\label{sec:uniform}

The uniform distribution is the simplest probability distribution,
assigning the same probability to every value of $x \in (a,b)$,
\begin{equation}
  \label{eq:uniformdistro}
  p(x) =  
\begin{cases}
    \frac{1}{b-a} & \text{ if $a < x < b$} \\
    0 & \text{otherwise}
  \end{cases}
\end{equation}
\begin{figure}
  \centering
  \begin{tikzpicture}
    \begin{axis} [
    width=\linewidth, 
    height=2in,
    domain=-1:5,
    samples=200,
    xmin=-1, xmax=5
    ]
    \addplot gnuplot[id=uni, mark=none, muted-green, ultra thick]{
    	     x<1 ? 0 : x>4 ? 0 : 0.25
    };
    \end{axis}
\end{tikzpicture}
  \caption{A uniform distribution}
  \label{fig:uniform}
\end{figure}
its accompanying cumulative distribution is
\begin{equation}
  \label{eq:uniformcdf}
P(X<x) = 
\begin{cases}
  0 & x' < a \\
\frac{x'a}{b-a} & a < x' < b \\
1 &  x' \geq b
\end{cases}
\end{equation}

\subsection{Gaussian Distribution}
\label{sec:gaussian}

\begin{figure}
  \centering
  \input{figures/gaussian.pgf}
\label{fig:gaussian}
  \caption{The Gaussian, central, or normal distribution.}
\end{figure}

This has the distribution function 
\begin{equation}
  \label{eq:gaussian}
  p(x) = \frac{1}{\sqrt{2 \pi} \sigma} \exp \qty(- \frac{(x-\mu)^2}{2 \sigma^2})
\end{equation}
The CDF has no analytic form, but when a special function, $\erf(t)$
is introduced a sei-analytic one can be found.
\begin{equation}
  \label{eq:cdfgaussian}
  \Phi(t) = \frac{1}{\sqrt{2 \pi} \sigma} \int_{-\infty}^t \exp \qty(- \frac{(x-\mu)^2}{2 \sigma^2}) \dd{x}
\end{equation}
\begin{equation}
  \label{eq:erf}
  \erf(t) = \frac{2}{\sqrt{\pi}} \int_0^t e^{-x^2} \dd{x}
\end{equation}
The Gaussian distribution is important, as, according to the central
limit theorem, other pdfs tend towards the central distribution as the
number of samples approaches $\infty$, and because it is the correct
pdf to use when only a mean and a variance are known for a stochastic
function.

\subsection{The $\chi^2$ Distribution}
\label{sec:chi2-distribution}

Consider a single observation taken from a Gaussian likelihood,
\[ p(x | \mu, \sigma) = \frac{1}{\sigma \sqrt{2 \pi}} \exp( -
\frac{(x-\mu)^2}{2 \sigma^2} ) \]
with $k$ observations this becomes
\[ p(\set{x_i} | \set{\mu_i}, \set{\sigma_i}) = \prod_{i=1}^k
\frac{1}{\sigma_i \sqrt{2}\pi} \exp( - \frac{(x_i - \mu_i)^2}{2
  \sigma_i^2} )\]
and so the log likelihood is
\[ \log(p) \propto - \sum_{i=1}^k \frac{(x_i - \mu_i)^2}{2
  \sigma_i^2} \] where the quantity on the right hand side is a
statistic drawn from a $\chi^2$-distribution.

\begin{definition}[{$\chi^2$}-distribution]
  The probability density function for th $\chi^2$-distribution is described by
  \begin{equation}
    \label{eq:4}
    p(x | k) = \frac{1}{2^{\frac{k}{2}} \Gamma \qty (\frac{k}{2}) } x^{\frac{k}{2} -1} e^{- \frac{x}{2}} 
  \end{equation}
\end{definition}

\begin{figure}
  \centering
  \begin{tikzpicture}
    \begin{axis} [
    width=\linewidth, 
    height=2in,
    domain=-10:10,
    samples=200,
    xmin= 0.1, xmax=10,
    ymin=0, ymax=0.6
    ]
    \def\k{1}
    \addplot gnuplot[id=chisq, mark=none, accent-red, ultra thick, restrict y to domain=0:2]{
      2**(-0.5*\k)*(gamma(0.5*\k)**-1) * x**(0.5*\k - 1) * exp(-x/2)
    };
    \def\k{2}
    \addplot gnuplot[id=chisq2, mark=none, muted-orange, ultra thick, restrict y to domain=0:2]{
      2**(-0.5*\k)*(gamma(0.5*\k)**-1) * x**(0.5*\k - 1) * exp(-x/2)
    };
    \def\k{3}
    \addplot gnuplot[id=chisq3, mark=none, accent-green, ultra thick, restrict y to domain=0:2]{
      2**(-0.5*\k)*(gamma(0.5*\k)**-1) * x**(0.5*\k - 1) * exp(-x/2)
    };
    \def\k{4}
    \addplot gnuplot[id=chisq4, mark=none, accent-blue, ultra thick, restrict y to domain=0:2]{
      2**(-0.5*\k)*(gamma(0.5*\k)**-1) * x**(0.5*\k - 1) * exp(-x/2)
    };
    \end{axis}
\end{tikzpicture}
  \caption{The $\chi^2$ distribution for $k \in [1,4]]$.}
  \label{fig:chisquareddistropdf}
\end{figure}
The distribution is illustrated in figure
\ref{fig:chisquareddistropdf}. 

\subsection{Determining PDFs}
\label{sec:determinepdf}

The most intuituve way to determine is to make repeated measurements
of a stochastic function to produce a histogram of the results. Then,
arranging these into a correctly normalised histogram, as the number
of trials and bins approaches infinity the histogram will approach the
PDF.\\
In a Bayesian perspective, the PDF represents the state of knowledge
about the noise: a uniform PDF has no preference for one value over
another; a central pdf represents knowledge of only its mean and
variance, and so on.

\subsection{Central Limit Theorem}
\label{sec:centrallimit}

Considering a Poisson distribution,
\[ p(N) = \exp(-\lambda) \frac{\lambda^N}{N!} \]
then, for $N\gg1$, it can be shown that 
\[ p(N) \approx \frac{1}{\sqrt{2 \pi \lambda}} \exp\qty(- \frac{(N-\lambda)^2}{2 \lambda}) \] which is the Gaussian distribution, with $\mu = \lambda$ and $\sigma = \sqrt{\lambda}$.

\subsection{Moments of a Distribution}
\label{sec:moments}

The $r^{\rm th}$ moment of a pdf\footnote{note that not all pdfs have defined moments, for example the Cauchy distribution, the integral of which diverges.}, $p(x)$, is defined as 
\begin{equation}
  \label{eq:moment}
  \langle x^r \rangle = \int_{-\infty}^{\infty} x^r p(x) \dd{x}
\end{equation}
The \emph{first moment} of the stochastic function is
\[ \langle x \rangle = \int_{-\infty}^{\infty} x p(x) \dd{x} \] This
is the {\em mean} of the stochastic function, or the expectation
value, $E(X)$, of $X$.\\
The \emph{second moment} is
\[ \langle x^2 \rangle = \int_{-\infty}^{\infty} x^2 p(x) \dd{x} \]
which is the mean square of the stochastic function. \\ The variance is defined in terms of this moment,
\begin{align}
  \label{eq:variance}
  \sigma^2 = {\rm var}(x) &= \int_{-\infty}^{\infty} \qty( x - \langle x \rangle)^2 p(x) \dd{x} \\
          &= \ev{\qty(x - \langle x \rangle )^2 } \nonumber \\
          &= \ev{x^2} - \ev{2x \ev{x}} + \ev{x}^2  \nonumber\\
          &= \ev{x^2} - 2 \ev{x} \ev{x} + \ev{x}^2 \nonumber\\
          &= \ev{x^2} = \ev{x}^2
\end{align}
which is the mean square deviation from the mean.
The {\em standard deviation}, $\sigma$, is the square-root of the variance.

\subsection{Other Measures of Distributions}
\label{sec:measures}

The \emph{median} of a stochastic function divides the CDF into two halves, such that
\begin{equation}
  \label{eq:median}
  \int_{-\infty}^{x_{\rm med}} p(x) \dd{x} = 0.5
\end{equation}
The \emph{mode} corresponds to the most pobable value of a stochastic function, if such a single value exists.

\subsection{Changing Variable}
\label{sec:changevariable}

In teh case that we know the distribution for a variable, $x$, but
want to find the corresponding distribution for a variable $y(x)$ we
need a change of variable. We know that the probability contained in
equivalent intervals must be identical. Assuming that $y$ is a monotonic function, for a small interval, 
\[ p(x) \qty| \delta x | = p(y) \qty| \delta y | \]
and in the limit
\begin{equation}
  \label{eq:changevaronedim}
  p(y) = p(x) \qty| \frac{\dd{x}}{\dd{y}} |
\end{equation}
If $y$ is \emph{not} monotonic, we need to be more careful.

In more than one dimension we must turn to the Jacobian, so
\[ p(y_1, y_2, \dots, y_n) = \qty|J| p(x_1, x_2, \dots, x_n) \]
with
\begin{equation}
  \label{eq:jacobian}
  J = \pdv{(x_1, x_2, \dots, x_n)}{(y_1, y_2, \dots, y_n)}
\end{equation}
which is a matrix with elements
\[ J_{ij} = \pdv{x_i}{y_i} \]

\begin{example}{\em The Exponential Distribution}\\
  The random number generator in a programming library will normall
  provide a uniform distribution between 0 and 1. That is
\[ p(x) \dd{x} = 
\begin{cases}
  \dd{x} & \text{ if } 0 < x < 1 \\
0 & \text{ otherwise. }
\end{cases}\] However, if we wish to generate an exponential
distribution we would choose
\[ y(x) = - \log(x) \] such that \[ x = e^{-y(x)} \]
Then
\[ p(y) \dd{y} = p(x) \abs{\dv{x}{y}} \dd{y} \]
and thus
\[ p(y) = \exp(-y) \] So $y = - \log(x)$ will provide an exponential
distribution.
\end{example}

\subsection{Multivariate Distributions}
\label{sec:mult-distr}

The concept of a probability distribution extends easily into higher
dimensions. We then get joint probability distributions, where there
are two variables involved with the probability of an event: the joint
probability of two events, $x_1$ and $x_2$ can be described
\begin{equation*}
  p(a_1 \le x_1 b_1 \text{ and } a_2 \le x_2 \le b_2) = \int_{a_1}^{b_1} \int_{a_2}^{b_2} p(x_1, x_2) \dd{x_1} \dd{x_2}
\end{equation*}
and this continues to higher dimensions.  The {\em marginal
  distribution} of a variable within a joint distribution is the
probability distribution for just that variable, and is found by
integrating over the others.
\begin{equation}
  \label{eq:marginal}
  p(x_1) = \int_{-\infty}^{\infty} p(x_1, x_2) \dd{x_2}
\end{equation}
Two variables are then {\em statistically independent} iff their joint
PDF can be written as the product of the marginal PDFs
\begin{equation}
  \label{eq:independentvariables}
  p(x_1, x_2) = p_1(x_1) p_2(x_2)
\end{equation}
put another way, if $x_1$ and $x_2$ are independent, 
\[ p_1(x_1) = p(x_1|x_2) \]

\subsection{The Bivariate Normal Distribution}
\label{sec:bivar-norm-distr}

The bivariate normal distribution is a joint distribution of two
independent variables, $x_1$ and $x_2$ with the form
\begin{equation}
  \label{eq:bivariatenormal}
  p(x_1, x_2) = \frac{1}{2 \pi \sigma_1 \sigma_2} \exp( - \frac{(x_1 - \mu_1)^2}{2 \sigma_1^2} - \frac{(x_2 - \mu_2)^2}{2 \sigma_2^2} )
\end{equation}
which can be simplified somewhat by letting
\[ \vec{x} = (x_1, x_2) \] \[ \vec{\mu} = (\mu_1, \mu_2) \] and 
\[ A = \begin{pmatrix}  \frac{1}{\sigma_1^2} & 0\\ 0 & \frac{1}{\sigma_2^2} \end{pmatrix}\]
so
\begin{equation}
  \label{eq:binormal2}
  p(\vec{x}) = \frac{1}{2 \pi \sigma_1 \sigma_2} \exp( \half (\vec{x} - \vec{\mu}) A (\vec{x} - \vec{\mu})^{\rm T})
\end{equation}

Note that 
\[ A^{-1} =
\begin{pmatrix}
  \sigma_1^2 & 0 \\ 0 & \sigma_2^2
\end{pmatrix} \]
so letting $C = A^{-1}$, we find a framework for a more general distribution,
\[ p(\vec{x}) = \frac{1}{2 \pi ({\rm det}(C)^{\half}} \exp( - \half ( \vec{x} - \vec{\mu})C^{-1} (\vec{x} - \vec{\mu})^{\rm T}) \]

$C$ is called the covariance matrix of the variables. If the variables
are independent it will be diagonal, however, it is possible to extend
the definition to incorporate correlated variables, where it takes on the form
\[ C =
\begin{pmatrix}
  \sigma_{11} & \sigma_{12} \\ \sigma_{21} & \sigma_{22}
\end{pmatrix}
\]
It is also possible to define the correlation coefficient between two variables, 
\begin{equation}
  \label{eq:corrco}
  \rho_{12} = \frac{\sigma_{12}}{\sigma_1 \sigma_2}
\end{equation}
The general bivariate normal distribution is then
\begin{equation}
  \label{eq:genbivariatenormal}
  p(x,y) = \frac{1}{2 \pi \sigma_x \sigma_y \sqrt{1-\rho^2}} \exp( - \frac{1}{2(1-\rho^2)} Q(x,y) )
\end{equation}
with
\[ Q(x,y) = \qty( \frac{x - \mu_x}{\sigma_x} )^2 - 2 \rho \qty(
\frac{x-\mu_x}{\sigma_x}) \qty( \frac{y - \mu_y}{\sigma_y}) + \qty(
\frac{y - \mu_y}{\sigma_y})^2 \] with the marginal distributions of
each variable being simply a univariate normal distribution.

\begin{figure*}
\begin{tikzpicture} 
\def\number{3}
\pgfmathsetmacro{\frac}{\textwidth/\number}
\matrix[column sep=0.5cm, row sep=0.5cm]{
  \def\rhov{0.0}
    \begin{axis}[view={0}{90}, domain=-.25:.25, width=\frac, height=\frac, xtick=\empty, ytick=\empty,
                 title={$\rho=0.0$}
                ]
      \addplot3[contour gnuplot={number=10,  labels=false }, ultra thick] 
               {exp(x^2 - 2* (\rhov)*x*y +y^2)}; 
      \addplot [dotted] {(\rhov)*x};
    \end{axis} 
&
  \def\rhov{0.2}
    \begin{axis}[view={0}{90}, domain=-.25:.25, width=\frac, height=\frac, xtick=\empty, ytick=\empty,
                 title={$\rho=\rhov$}
                ]
      \addplot3[contour gnuplot={number=10,  labels=false }, ultra thick] 
               {exp(x^2 - 2* (\rhov)*x*y +y^2)}; 
      \addplot [dotted] {(\rhov)*x};
    \end{axis} 
&
  \def\rhov{0.4}
    \begin{axis}[view={0}{90}, domain=-.25:.25, width=\frac, height=\frac, xtick=\empty, ytick=\empty,                 title={$\rho=\rhov$}]
      \addplot3[contour gnuplot={number=10,  labels=false }, ultra thick] 
               {exp(x^2 - 2* (\rhov)*x*y +y^2)}; 
      \addplot [dotted] {(\rhov)*x};
    \end{axis}
\\
  \def\rhov{0.6}
    \begin{axis}[view={0}{90}, domain=-.25:.25, width=\frac, height=\frac, xtick=\empty, ytick=\empty,                  title={$\rho=\rhov$}]
      \addplot3[contour gnuplot={number=10,  labels=false }, ultra thick] 
               {exp(x^2 - 2* (\rhov)*x*y +y^2)}; 
      \addplot [dotted] {(\rhov)*x};
    \end{axis} 
&
  \def\rhov{0.8}
    \begin{axis}[view={0}{90}, domain=-.25:.25, width=\frac, height=\frac, xtick=\empty, ytick=\empty,                  title={$\rho=\rhov$}]
      \addplot3[contour gnuplot={number=10,  labels=false }, ultra thick] 
               {exp(x^2 - 2* (\rhov)*x*y +y^2)}; 
      \addplot [dotted] {(\rhov)*x};
    \end{axis}
&
  \def\rhov{1}
    \begin{axis}[view={0}{90}, domain=-.25:.25, width=\frac, height=\frac, xtick=\empty, ytick=\empty,                  title={$\rho=\rhov$}]
      \addplot3[contour gnuplot={number=10,  labels=false }, ultra thick] 
               {exp(x^2 - 2* (\rhov)*x*y +y^2)}; 
      \addplot [dotted] {(\rhov)*x};
    \end{axis}   
\\
};
\end{tikzpicture}
\caption{The Bivariate Normal Distribution, plotted at various values
  of $\rho$, alongside the regression line of $y$ on $x$.}
\label{fig:bivariatenormal}
\end{figure*}

\subsection{Samples and Parents}
\label{sec:samples-parents}

The quantities, such as ``mean'', ``standard deviation'', and
``covariance'' define the PDF of a given quantity. But when these are
found from a finite set of data they represent only an approximation
of the true quantities defining the true distribution of the
data. Such quantities are the ``sample mean'' and the ``sample
variance''. The true parameters are known as the ``distribution mean''
or the ``parent mean''.

\section{Parameter Estimation}
\label{sec:estimationpar}

Often we are presented with the task of estimating something using
data that contain random errors or other uncertainties. The rest of
the course is on how to do that well.

Imagine we take $n$ independent measurements of some quantity under
identical conditions (or so we believe). We can compute the sample
mean, $\bar{x}$, and the variance, $\sigma^2$, of our measurements:
\[ \bar{x} = \frac{1}{n} \sum x_i \]
\[ \sigma^2 = \frac{1}{n} \sum (x_i - \bar{x})^2 \] Here $\sigma$ is
our estimate of the error on each measurement, and we average $n$ of
them, so we expect the sample mean to deviate from the true mean by
only around $\frac{\sigma}{\sqrt{n}}$, i.e\
\[ \mu = \bar{x} \pm \frac{\sigma}{\sqrt{n}} \] so we would expect
$\mu$ to lie in this range 68\% of the time.  i.e.\ \[P \qty(\bar{x}-
\frac{\sigma}{\sqrt{n}} \le \mu \le \bar{x} + \frac{\sigma}{\sqrt{n}})
= 68\% \]

\subsection{A General Procedure for Parameter Estimation}
\label{sec:genproc}

In a general problem we have a model of how the world works which
depends on a number of parameters (e.g. binomial). The model tells us
the likelihood of any given set of data, $\{D_k\}$, given a value for
the parameters, $L = p(\{D_k\} | m,I)$. We can take into account prior
probability of the model based on prior knowledge, ${\rm Prior} =
p(m|I)$. Bayes' Theorem provides us with the posterior probability of
the model parameters.
\[ {\rm posterior} = p(m|\{D_k\}, I) \propto p(m|I) p(\{D_k\}|m,I) \]
which we can then normalise. This is all that BPT has to say about the
value of $m$.\\
We will often want to summarise this in terms of two numbers; the best
estimate, and the uncertainty.

\subsubsection{Best estimates and error bars}
\label{sec:bestestimate}

The posterior pdf for some model parameter $x$ can have any shape, but
it is usually peaked. The most probable value of $x$ is then $x_0$.
\[ p(x_0 | {\rm data}) = p_{\rm max} \]
and
\[ \eval{\dv{p}{x}}_{x_0} = 0 \qquad \eval{\dv[2]{p}{x}}_{x_0} <0 \]
We could expand a Taylor series about $x_0$, but we usually need fewer terms for the same accuracy using
\[ L = \log \qty[ p(x|{\rm data})] \]
the log-prior probability, since this has a critical point at $x_0$ too.
Expanding $L$ about $x_0$, 
\[ L(x) = L(x_0) + \half \eval{ \dv[2]{L}{x} }_{x_0} (x - x_0)^2 + \cdots \]
By stopping at the second term we have
\[ p(x|{\rm data}) \approx c \exp( \half \eval{\dv[2]{L}{x}}_{x_0} (x-x_0)^2 ) \]
noting that the second derivative is negative, this is a Gaussian PDF,
\[ p(x|{\rm data}) \approx \frac{1}{\sqrt{2 \pi} \sigma} \exp( -
\frac{(x-x_0)^2}{2 \sigma^2} ) \] with $\sigma = \sqrt{- \eval{
    \dv[2]{L}{x} }_{x_0} }$.

In a Gaussian distribution, around $67\%$ of the probability lies
withing $\sigma$ of $x_0$, so
\[ p(x_0 - \sigma \le x \le x_0 + \sigma | {\rm data}) = 0.67 \]

\begin{example}
  The Gaussian approximation can be applied to the coin example.
  \[ p(H|\{D_k\},I) \propto H^R (1-H)^{N-R} \] which is the posterior
  probability of $H$ given $R$ heads and $N$ tosses using a uniform
  prior.
  \[ L = \log(p) = R \log(H) + (N-R) \log(1-H) + {\rm const.} \]
so
\[ \dv{L}{H} = \frac{R}{H} - \frac{(N-R)}{(1-H)} \]
and
\[ \dv[2]{L}{H} = - \frac{R}{H^2} - \frac{(N-R)}{(1-H)^2} \] The most
probable $H$, $H_0$, occurs when \[ -\eval{ \dv{L}{H} }_{H_0} = 0 \]
that is,
\[ \frac{R}{H_0} - \frac{(N-R)}{(1-H_0)} = 0 \]
so
\[ H_0 = \frac{R}{N} \]
which is just the relative frequency of heads.
We also have
\[ \sigma \approx \qty( - \eval{ \dv[2]{L}{H} }_{H_0} )^{-\half} \]
thus
\[ \sigma = \qty( \frac{H_0(1-H_0)}{N} )^{\half} \] For a fixed $N$ we
are most sure of the value of $H$ when $H_0 \approx 0$ and $H_0
\approx 1$, as it's very easy to spot when a coin is extremely
biased. For a fixed $N$, $\sigma$ is greatest when $H_0=0.5$, so it's
hard to be confident that a coin is fair. The peak of the posterior,
$p(H_0)$ doesn't shift much after a few flips, so from that point on,
\[ \sigma \propto \frac{1}{\sqrt{N}} \] so the uncertainty in $H$
drops as $\frac{1}{\sqrt{N}}$.
\end{example}

\subsection{Shortest Confidence Intervals}
\label{sec:short-conf-interv}

Defining a confindence interval for a symmetric PDF is straightforward
and unique process, of finding the area corresponding to a given
quantity of the total probability. In an asymmetric PDF the choice is
less clear, as there may be many choices of areas containing a given
percentage of the proability. To get around this problem we require a
shortest confidence interval, that is
\[ p(x_1 \le x \le x_2 | \{ \rm data \}, I) = 0.95 \]
and 
\[ \abs{x_2 - x_1} \] is a minimum. In the case of an asymmetric PDF
the value of $x_0$ is not the expectation value of $x$, but instead
the most probable value given the data.

\subsection{Gaussian Noise}
\label{sec:gaussian-noise}

It is common for a set of direct measurements of some given parameter
to be contaminated by gaussian noise. For example the flux density of
a Gaussian source, or measurements of the mass of the Sun. In this case, let the true value of the parameter be $\mu$, and let the $k$th measurement of it be 
\[ x_k = \mu + {\rm noise}_k \] If we assume the noise is Gaussian,
with a mean of $0$ and a variance $\sigma_0^2$,
\[ {\rm noise}_k = x_k - \mu \]
Now, given $\mu$, the probability of making a measurement $x_k$ is just the probability that the residual, $x_k-\mu$ can be attributed to the noise, that is
\[ p(x_k | \mu, \sigma_0, I) = \frac{1}{\sigma \sqrt{2 \pi}} \exp( -
\frac{(x_k-\mu)^2}{2 \sigma_0^2} )\] If the $N$ measurements, $\{ x_k
\}$ are independent, then their joint probability (likelihood) is
\begin{align*} 
p( \{ x_k \} | \mu, \sigma_0, I ) &= \prod_{k=1}^N p(x_k | \mu, \sigma_0, I) \\
&= \frac{1}{\sigma_0^N (2 \pi)^{\frac{N}{2}}} \exp( - \sum_{i=1}^N \frac{(x_k - \mu)^2}{2 \sigma_0^2} )
\end{align*}
In order to turn this into a posterior probability for $\mu$ given the
data, and $\sigma_0$ we need Bayes Theorem, and a prior on $\mu$. Let
us choose a uniform prior in the range $[\mu_{\rm min}, \mu_{\rm
  max}]$, so
\[ p(\mu|\sigma_0, I) =
\begin{cases}
  \frac{1}{\mu_{\rm max} - \mu_{\rm min}} & \mu_{\rm min} \le \mu \le \mu_{\rm max} \\
0 & \text{ otherwise }
\end{cases}
\]
Now,
\[ p(\mu | \{ \rm data \}, \sigma_0, I) \propto p(\mu | \sigma_0, I) \cdot p( \{ x_k \} | \mu, \sigma_0, I ) \]
and
\[ \log[ p(\mu | \{ x_k \}, \sigma_0, I) = L = - \sum_{k=1}^N \frac{(x_k - \mu)^2}{2 \sigma_0^2} + {\rm const}] \]
The value of $\mu$ which maximises $L$ is its most probable value (given the data and $\sigma_0$), $\mu_0$, i.e.\ 
\[ \eval{ \dv{L}{\mu} }_{\mu_0} = \sum_{k=1}^N \frac{(x_k - \mu_0)}{\sigma_0^2} = 0 \]
as $\sigma^2$ is constant,
\[ \sum x_k - \sum \mu_0 = 0 \]
\[ \therefore \sum x_k = \sum \mu_0 \]
So the most probable value of the parameter is simply the mean of the measurements,
\[ \mu_0 = \frac{1}{N} \sum^N_{k=1} x_k \]

The standard confidence interval is defined by
\[ \qty( - \eval{\dv[2]{L}{\mu}}_{\mu_0} )^{\half} = \qty(
\sum_{k=1}^N \frac{1}{\sigma_0^2})^{- \half} = \qty(
\frac{N}{\sigma_0^2} )^{-\half} = \frac{\sigma_0}{\sqrt{N}} \]
So the posterior pdf for $\mu$ can be summarised as 
\[ \mu = \mu_0 \pm \frac{\sigma_0}{\sqrt{N}} \]

\subsection{Data with varying error bar sizes}
\label{sec:data-with-varying}

Thus far consideration has only been made of data which has constant
error bars, but in many cases each datum will have a specific error
bar. If the errors are still Gaussian we can proceed as before, but
each measurement, $x_k$ has an associated $\sigma_k$, now,
\[ L = - \sum_{k=1}^N \frac{(x_k - \mu)^2}{2 \sigma_k^2} + {\rm const.}\]
The most probable $\mu$, $\mu_0$ satisfies
\[ \eval{\dv{L}{\mu}}_{\mu_0} = - \sum_{k=1}^N \frac{(x_k -
  \mu_0)}{\sigma_k^2} = 0 \]
So
\[ \mu_0 = \frac{\sum_k \frac{x_k}{\sigma_k^2}}{ \sum_k \frac{1}{\sigma_k^2}} \]

This is a weighted mean; each datum has been loaded by our confidence
in it; the standard width of our posterior for $\mu$ can be calculated
as
\[ {\rm error in } \mu = \qty( - \eval{\dv[2]{L}{\mu}}_{\mu_0} )^{-
  \half} = \qty( \sum_{k=1}^N \frac{1}{\sigma_k^2} )^{- \half} \]

\subsection{Model Fitting}
\label{sec:model-fitting}

This approach is easily extended to more sophiticated problems.
\begin{example} {\em A Neutron Star}\\
  Consider a binary neutron star system which can be expected to
  produce sinusoidal gravitational waves at twice the system's orbital
  velocity. This can be modeled as
  \[ h(t) = A \sin(\omega t + \phi) \] We wish to determine the
  amplitude of the waves, $A$, from the output of a detector,
  $d(t_k)$. For this example assume that the noise is Gaussian (which
  is very debatable in a GW detector!), with a known variance
  $\sigma_0^2$. We proceed as before, but now the likelihood of the
  data depends on how well it fits the model.
  \[ p(\{ d_k \} | A, w, \phi) \propto \prod_k \exp( - \frac{(d_k - A
    \sin(\omega t_k + \phi))^2 }{\sigma_0^2} ) \]
We want to know the posterior for $A$,
\[ p(A | \{ d_k \}, \omega, \phi) \propto p(A | \omega, \phi) p(\{ d_k
\} | A, \omega, \phi) \] again, taking the prior as uniform for $A \ge
0$, and $0$ for $A < 0$, then,
\[ L = - \sum_{k=1}^N (\frac{d_k - A \sin(\omega t_k + \phi) )^2}{2 \sigma_0^2} \]
Our most probable $A$, $A_0$, satisfies
\[ \eval{\dv{L}{A}}_{A_0} = 0 \]
\[ A_0 = \frac{\sum_k d_k \sin(\omega t_k + \phi)}{\sum_k \sin[2](\omega t_k + \phi)} \]
\end{example}

\subsubsection{Handling multiple parameters}
\label{sec:handl-mult-param}

If in the previous example we neither knew $A$ nor $\phi$ we would be
able to determine a joint posterior for $A$ and $\phi$,
\[ p(A, \phi | \{ d_k \}, \omega) p ( \{ d_k \} | A, \omega, \phi ) \]
This represents a probability surface, which can be represented in a
contour map. The most probable value of $A$ is $A_0$, and $\phi$ is
$\phi_0$. We have used the prior,
\[ p(A, \phi | \omega) = p(A|\omega) p(\phi|\omega) \]
and 
\begin{align*}
  p(A|\omega) &= {\rm const}, A>0, 0 {\rm otherwise} \\
p(\phi | \omega) &= {\rm const}, -\pi < \phi < \pi, 0 {\rm otherwise}
\end{align*}

\subsection{Marginal Distributions}
\label{sec:marg-distr}

Suppose we are not interested in $\phi_0$, then we can determine the
posterior pdf for $A$ alone if we marginalise over $\phi$, so
\[ p(A | \{ d_k \}, \omega) = \int_{-\infty}^{\infty} p(A, \phi | \{
d_k \}, \omega) \dd{\phi} \] We have projected the 2D posterior onto
the $A$-axis. The value of $A_0$ from the marginal distribution need
not be the same as the one derived from the joint distribution.

\begin{example}{\em Estimating two parameters simultaneously}\\
   Consider the observation of an optical spectral line using a good
   CCD. Within the data a spectral line is present, and we want to
   estimate its amplitude, $A$, which will be represented in the data
   by a data number $A n_0$, for $n_0$ being a constant depending on
   the detector efficiency and exposure time. There is also a
   background level of data, arising from the atmosphere and the CCD's
   dark current, which has an amplitude $B n_0$.\\
   Given the shape (but not the height) of the spectral line, how do we
   optimally use the data to estimate $A$?\\
   In order to do this we perform a fit consistent with the noise,
   combining all the available information in the data. We also know
   the line to be thermally broadened, so it has a Gaussian profile
   with a wifth $w$. The $k$th datum of the ideal (green) line is therefore
   \[ D_k = n_0 \qty[ A \exp(- \frac{(x_k -x_0)^2}{2w^2} + B ) ] \] with
   $x_k$ the position of the $k$th datum, and $x_0$ the position of the
   centre of the line. The $\{ D_k \}$ values comprise the model; we
   imagine that we are told $n_0$, $x_0$, and $w$, so our job is to
   estimate $A$ from the real data.\\ 
   We are counting photons, and we expect to find $D_k$ photons in the
   $k$th pixel on average. The probability of getting $N_k$ photons is
   \[ p(N_k | D_k) \frac{D_k^{N_k} \exp(-D_k)}{N!} \] which is the
   Poisson distribution, describing the probability of the data given
   our model, that is, the likelihood of the data. Noting that $D_k$
   depends on the model parameters, $A$ and $B$, the joint likelihood
   of all the data, taken as independent, is
   \[ p( \{ N_k \} | A, B, I) = \prod_{k=1}^m p(N_k | A,B,I) \] In
   order to find a posterior for $A$ and $B$ we need to set priors on
   $A$ and $B$. Let's choose 
   \[ p(A,B|I) =
   \begin{cases}
     \text{constant for} & A>0, B>0\\
     0 & \text{otherwise}
   \end{cases}
   \]
\end{example}

\subsection{A Shortcut for Maximum Likelihood}
\label{sec:shortc-maxim-likel}

The correct Bayesian approach is always to calculate the posterior
distribution of the parameter being estimated, but the prior is often
approimately constant over the area which contains the most
likelihood. When this is the case the posterior will be directly
proportional to the likelihood, and the most probable $x$ will be the
value which maximises the likelihood of the data.

\subsection{Least Squares}
\label{sec:least-squares}

We can go further, and assume our data is affected by independent Gaussian errors, then, as before,
\[ L \propto \exp( - \sum_{k=1}^N \frac{(F_k - D_k)^2}{\sigma_k^2}
) \] for a model $F_k$, data $D_k$, and variance of the $k$th datum
$\sigma_k^2$. The most probable value of the model parameter is then the one which minimises
\[ \chi^2 = \sum_{k=1}^N \frac{(F_k - D_k)^2}{\sigma_k^2} \] If the
errors on all of the data are the same then we need only minimise
\[ \sum_{k=1}^N (F_k - D_k)^2 \] that is, find the least square
difference between the model and the data summed over all of the data.

\subsubsection{Straight-line fitting with least squares}
\label{sec:stra-line-fitt}

A classic eample of the least-squares approach to regression is
fitting a linear model, of the form $y = m +c$ to data. If we have
errors only in the $y$ component, so that $\sigma_k$ is the error on
the $k$th datum, then the data are $\{ D_k \}$ at $x$-positions $\{
x_k \}$, so that
\[ \chi^2 = \sum^N_{k=1} \frac{(m x_k + c - D_k)^2}{\sigma_k^2} \]
Then, $\chi^2$ has a minumum when both
\[ \pdv{\chi^2}{m} = 0 \quad \text{and} \quad \pdv{\chi^2}{c}=0 \]
so,
\[ m \sum \frac{x_k^2}{\sigma_k^2} + c \sum \frac{x_k}{\sigma^2_k} = \sum \frac{D_k x_k}{\sigma_k} \]
and
\[ m \sum \frac{x_k}{\sigma_k^2} + c \sum \frac{1}{\sigma_k^2} = \sum
\frac{D_k}{\sigma_k^2} \] then what remains is to solve for $m$ and
$c$.

\subsection{Evidence}
\label{sec:evidence}

Thus far we have concentrated on parameter estimation based on data,
$d$, using Bayes' Theorem:
\[ p(a|d,I) = \frac{p(a,I) p(d|a,I)}{p(d|I)} \]

Because we have only been interested in proportionality it has been
possible to ignore the denominator from the right hand side, as it
only affects the normalisation of $p(a|d,I)$. The quantity $p(d,I)$ is
the evidence, or marginal likelihood, with
\begin{align*}
  p(d|I) &= \int p(d, a|I) \dd{a} \\ &= \int p(a|I) p(d|a,I) \dd{a} 
\end{align*}

It represents the joint probability of the data and the parameter
which is marginalised over; the probability of the parameter
regardless of the value of the parameter. It is thus the likelihood of
the model which parameterises the problem with a parameter $a$. The
probability of the model would be
\[ p(I|d) = \frac{p(I) p(d|I)}{p(d)} \] We are now considering the
probabilities of models rather than their parameters, giving a rather
vague measure, however, a well-chosen model should have a relatively
high evidence value.

\begin{example}{\em Two different models fitted to data}\\
  Suppose some data are fitted to a model of the form
  \[A \sin(\omega t_i) \] such that
\[ p(A, \omega | \{ d_i \}, I_1) \propto p(A, \omega) \prod_i \exp( - \frac{\qty(d_i - A \sin(\omega t_i))^2}{2 \sigma^2} ) \]
but are also fitted to a model of the form
\[ A \exp( - \frac{t}{\tau} ) \] such that
\[ p(A,\tau | \{ d_i \}, I_2) \propto p(A, \tau) \prod_i \exp( -
\frac{\qty(d_i - A \exp( - \frac{t_i}{\tau}) )^2}{2 \sigma^2} ) \]
Both will produce a fit, but to establish which gives the better fit
we can turn to the evidence, and we find, all other things being
equal, that \[ p( \{ d_i \} | I_1) \gg p( \{ d_i \} | I_2) \]
\end{example}

\subsection{Parameter Estimation in Two Dimensions}
\label{sec:param-estim-two}

Consider the form of the bivariate normal distribution in equation
(\ref{eq:genbivariatenormal}), where there is a quantity $Q$. We can
re-express $Q$ as a matrix,
\begin{equation}
  \label{eq:1}
  Q =
  \begin{pmatrix}
    X-X_0 & Y-Y_0
  \end{pmatrix}
  \begin{pmatrix}
    A & C \\ C & B
  \end{pmatrix}
  \begin{pmatrix}
    X-X_0 \\ Y-Y_0
  \end{pmatrix}
\end{equation}
where
\begin{align*}
  A &= - \eval{ L_{xx} }_{X=X_0} \\
  B &= - \eval{ L_{yy} }_{Y=Y_0} \\
  C &= - \eval{ L_{xy} }_{X=X_0, Y=Y_0}
\end{align*}
Assuming that our distribution is Gaussian,
\begin{align*}
  A^{-1} &= \sigma_X^2 \\
  B^{-1} &= \sigma_Y^2 \\
  C^{-1} &= \sigma_{XY}^2
\end{align*}
then
\begin{align*}
  \begin{pmatrix}
    \sigma_X^2 & \sigma_{XY}^2 \\ \sigma_{XY}^2 & \sigma_Y^2
  \end{pmatrix} &=
-
\begin{pmatrix}
  A & C \\ C & B
\end{pmatrix}^{-1} \\
&= \frac{1}{AB - C^2}
\begin{pmatrix}
  - B & C \\ C & -A
\end{pmatrix}
\end{align*}

This approach can be extended to more than two dimensions, where
\begin{equation}
  \label{eq:2}
  F = F_{ij} = \pdv{L(\theta_1, \theta_2, \dots, \theta_n)}{\theta_{ij}}
\end{equation}
where $F$ is the Fischer Information Matrix. If the matrix is diagonal,
\[ F = - \diag( \sigma_1^{-2}, \sigma_2^{-2}, \dots, \sigma_n^{-2}
) \]The Fischer information matrix provides a measure of how much
information a dataset can yield about the parameters of a model, as
the FIM is the inverse of the covariance matrix. In the cases where
$F$ is not diagonal it can aid us by revealing which combinations of
parameters are well constrained by the data.

\begin{example}
 In an experiment, the probability of the $k$th datum having a value $x_k$ is given by
\[ p(x_k | \mu, \sigma) = \frac{1}{\sigma \sqrt{2 \pi}} \exp( - \frac{(x_k-\mu)^2}{2 \sigma^2} )\]
where the error, $\sigma$, in each measurement is known. The posterior has the form

\[ p(\mu | \set{x_k}, \sigma, I) \propto p(\set{x_k} | \mu, \sigma, I)
p(\mu | \sigma, I ) \]

For the data set, containing $k$ measurements,
\begin{align*} 
p( \set{x_k} | \mu, \sigma,I) &= \prod_{i=1}^N p (x_k | \mu, \sigma, I) \\
&= \prod_{i=1}^N \frac{1}{\sigma \sqrt{2 \pi}} \exp[ - \frac{(x_k - \mu)^2}{2 \sigma^2}]
\end{align*}
then, assuming a uniform prior,
\begin{align*}
  p(\mu | \sigma, I) = p(\mu|I) =
  \begin{cases}
    a & \mu_{\rm min} \le \mu \le \mu_{\rm max} \\
0 & \text{otherwise}
  \end{cases}
\end{align*}
We then want the log of the posterior,
\begin{align*}
  L &= \log( p(\mu | \set{x_k}, \sigma, I) ) \\
&= \log(a) + \sum_{i=1}^N \log( p(x_k | \mu, \sigma, I) ) \\
&= \text{constant } - \sum_{i=1}^N \frac{(x_k - \mu)^2}{2 \sigma^2}
\end{align*}
Now, let $\mu_0$ be the best estimate of $\mu$, so
\begin{align*}
  \eval{ \dv{L}{\mu} } = \sum_{i=1}^N \frac{(x_k - \mu_0)}{\sigma^2} = 0
\end{align*}
From this,
\[ \sum_{i=1}^N x_k = \sum_{i=1}^N \mu_0 = N \mu_0 \]
or
\[ \mu_0 = \frac{1}{N} \sum_{i=1}^N x_k \]
Then, our confidence in the reliability of the best estimate is
\[ \eval{\pdv[2]{L}{\mu}}_{\mu_0} = - \sum_{i=1}^N \frac{1}{\sigma^2}
= - \frac{N}{\sigma^2} \]
So our inference of $\mu$ is summarised by
\[ \mu = \mu_0 \pm \frac{\sigma}{\sqrt{N}} \]
\end{example}

\begin{example}
  Consider a model of the form
  \[ y = a + bx \] where we want to estimate the parameters $a$ and
  $b$. Let us assume a broad and uniform prior on both parameters. All
  measurements are subject to Gaussian, independent noise with $\mu=0$
  and a variance $\sigma^2$. There are $n$ observations, $(\set{x_i}
  \times \set{y_i}), i \in [1,n]$.  Since we have uniform priors, the
  posterior is proportional to the likelihood, $l(a,b)$.
  \[ l(a,b) = \prod_{i=1}^n \frac{1}{\sqrt{2 \pi} \sigma} \exp[ -
  \frac{1}{2 \sigma^2} (y_i - a - bx_i)^2 ] \] We want to maximise the
  log likelihood, $L = log(l)$,
  \[ L(a,b) = - \frac{n}{2} \log(2 \pi) - n \log(\sigma) - \frac{1}{2
    \sigma^2} \sum_{i=1}^n (y_i - a - bx_i)^2 \] Then differenitating,
\begin{align*} 
\pdv{L}{a} &= \frac{1}{\sigma^2} \sum_{i=1}^n (y_i - a - bx_i) \\
\pdv{L}{b} &= \frac{1}{\sigma^2} \sum_{i=1}^n (y_i - a - bx_i)x_i
\end{align*}
The minimum occurs where these derivatives evaluate to zero, which give
\begin{align*}
  \sum_{i=1}^n y_i& = na + b \sum_{i=1}^n x_i \\
\sum_{i=1}^n x_i y_i &= a \sum_{i=1}^n x_i + b \sum_{i=1}^n x^2_i
\end{align*}
which can be rewritten as a matrix equation,
\begin{equation*}
  \begin{pmatrix}
    n & \sum x_i \\ \sum x_i & \sum x_i^2
  \end{pmatrix}
  \begin{pmatrix}
    a \\ b
  \end{pmatrix}
=
\begin{pmatrix}
  \sum y_i \\ \sum x_i y_i
\end{pmatrix}
\end{equation*}
This can be solved
\begin{align*}
  \begin{pmatrix}
    a \\ b
  \end{pmatrix} &=
  \begin{pmatrix}
    n & \sum x_i \\ \sum x_i & \sum x_i^2
  \end{pmatrix}^{-1}
  \begin{pmatrix}
    \sum y_i \\ \sum x_i y_i
  \end{pmatrix} \\
&= \frac{1}{n \sum x_i^2 - \qty( \sum x_i )^2}
\begin{pmatrix}
  \sum x_i^2 & - \sum x_i \\ - \sum x_i & n
\end{pmatrix}
\begin{pmatrix}
  \sum y_i \\ \sum x_i y_i
\end{pmatrix}
\end{align*}
from which we obtain
\begin{align*}
  a &= \frac{\sum x_i^2 \sum y_i - \sum x_i \sum x_i y_i}{n \sum x_i^2 - \qty( \sum x_i)^2} \\
b &= \frac{n \sum x_i y_i - \sum x_i \sum y_i}{n \sum x_i^2 - \qty( \sum x_i)^2}
\end{align*}
The FIM can then be computed from the expressions
\begin{align*}
  \pdv[2]{L}{a} &= \frac{n}{\sigma^2} \\
\pdv[2]{L}{b} &= \frac{\sum x_i^2}{\sigma^2} \\
\pdv[2]{L}{a}{b} &= \sum{x_i}{\sigma^2}
\end{align*}
So
\begin{equation*}
  F = \frac{1}{\sigma^2}
  \begin{pmatrix}
    -n & - \sum x_i \\ - \sum x_i & - \sum x_i^2
  \end{pmatrix}
\end{equation*}
thus
\begin{equation*}
  F^{-1} = \frac{\sigma^2}{n \sum x_i^2 - \qty( \sum x_i)^2} 
  \begin{pmatrix}
    \sum x_i^2 & - \sum x_i \\ - \sum x_i & n
  \end{pmatrix}
\end{equation*}
and thus
\begin{align*}
  \sigma^2_a &= \frac{\sigma^2 \sum x_i^2}{n \sum x_i^2 - \qty( \sum x_i)^2} \\
\sigma_b^2 &= \frac{\sigma^2 n}{n \sum x_i^2 - \qty( \sum x_i )^2} \\
\sigma_{ab}^2 &= \frac{- \sigma^2 \sum x_i}{n \sum x_i^2 - \qty( \sum x_i)^2}
\end{align*}
\end{example}

\section{Credible Regions}
\label{sec:credible-regions}

Consider the expession for log likelihood,
\[ L(\theta_1, \theta_2) = \text{constant} = \half \chi^2 (\theta_1,
\theta_2) \] we define $\chi^2 \equiv \chi^2_{\rm min}$ when
$(\theta_1, \theta_2) = (\theta_{01}, \theta_{02})$, and so $\Delta
\chi^2 = \chi^2 - \chi^2_{\rm min}$ which tells us the $\chi^2$ value
when the parameters' values which have been chosen do not correspond
to their true values. Thus to maximise the likelihood we minimise the
$\chi^2$.

\begin{align*}
  L(\theta_1, \theta_2) - L(\theta_{01}, \theta_{02}) &= - \half (\chi^2 - \chi^2_{\rm min}) \\ &= - \half \Delta \chi^2 \\
  L(\theta_1, \theta_2) &= L(\theta_{01}, \theta_{02}) - \half \Delta
  \chi^2
\end{align*}
So the posterior can then be expressed
\[ p(\theta_1, \theta_2 | \text{data}, I) = p(\theta_{01}, \theta_{02}
| \text{data}, I) \exp( - \half \Delta \chi^2) \] Now, each contour on
the posterior probability corresponds to a unique value of $\Delta
\chi^2$; the further the chosen parameters are from the minimising
value the broader the credible regions will be. These credible regions
are also referred to as confidence intervals.

\section{Principle Component Analysis}
\label{sec:princ-comp-analys}

A non-zero correlation coefficient produces ellipsoidal isoprobability
contours; to get around this we can re-parametrise the model to find
independent parameters, but we need to define new principle axes to do
this. To find independenet axes we need a diagonal Fisher Information
Matrix, and we can diagonalise to achieve one. \\
The characteristic equation is
\begin{equation}
  \label{eq:3}
  \begin{pmatrix}
    A & B \\ C & D
  \end{pmatrix}
  \begin{pmatrix}
    x \\ y
  \end{pmatrix}
= \lambda
\begin{pmatrix}
  x \\ y
\end{pmatrix}
\end{equation}
the eigenvectors will give the independent parameters, or principle
axes. This process allows multi-dimensional datasets to be converted
to data dependent on uncorrelated variables---the principle
components. The first principle component should be a linear
combination of the original variables which incorporates as much of
the cariation in the data as possible; the second principle component
should account for as much of the remaining variation as possible. In
addition, all of the principle components are mutually orthogonal.

\section{Markov Chain Monte Carlo Methods}
\label{sec:markov-chain-monte}

Consider a model $f(X)$ where $X$ represents the set of parameters for
the model which has a probability density
\[ p(X | D, I) \] for observed data $D$.
The expectation value for the model is then
\[ \ev{f(X)} = \int f(X) p(X|D,I) \dd{X} = \int g(X) \dd{X} \] if $X$
has a single element this is a straightforward computation, but with
many parameters we need to marginalise many times, which is
computationally intensive. For example, consider a Gaussian likelihood
with two parameters, $A$ and $B$. We can evaluate the function by
brute force over a grid, but we are only interested in the likelihood
around the peak of the function, so this is computationally
wasteful. Instead we can use (Monte Carlo) random sampling.

To do this we take random samples in the parameter space, but we need
to have the correct mean and standard deviation for the Gaussian from
which we draw the random points to take the samples. We can manage
this using a Markov chain technique, and Markov Chain Monte Carlo
methods are excellent for sampling high-dimension functions.

A Markov chain is a system which undergoes transitions from one state
to another on a state space, where the next state depends only on the
current state.

Suppose we have a likelihood function which depends on two variables,
$a$ and $b$, and we want to find the maximum.

\begin{algorithm}[Metropolis-Hastings Algorithm]\\
\begin{enumerate}
\item Start with $P_0 = (a, b) \gets (a_0, b_0)$ \\ i.e. select a
  point at random in the distribution.
\item Center a PDF, $Q$, called the proposal density, at this point.
\item Sample a new point, $P^{\prime}$ from $Q$.
\item Compute the ratio, $R$, of the likelihoods of the initial point
  and the new point.
\begin{itemize}
\item if $R>1$ the likelihood of $P^{\prime}$ is greater than at
  $P_0$, and so $P_1 \gets P^{\prime}$ as it becomes the next step in
  the chain.
\item else if $R<1$ we reject $P^{\prime}$ for the next point, and
  instead accept it with a probability $R$ by generating a a random number, $x ~ U[0,1]$.
  \begin{itemize}
  \item If $x<R$ accept $P^{\prime}$, so $P_1 \gets P^{\prime}$
  \item If $x>R$ reject P$^{\prime}$, so $P_1 \gets P_0$
  \end{itemize}
\end{itemize}
\end{enumerate}
\end{algorithm}

After running the Metropolis-Hastings algorithm we have a sequence of points 
\[ P_1, P_2, \dots, P_n \] which constitute a sample from the
likelihood function, this sequence generates a sequence for $a$ and
$b$. Making a histogram of these sequences we can estimate the mean
and variance of each parameter. The MH algorithm generates parameter
samples with a probability density which converges on the desired
target posterior, or the stationary distribution of the Markov Chain.

In order to converge on a stationary distribution the Markov chain
must be irreducible (i.e.\ must be able to jump to all states with
positive probability), aperiodic (i.e.\ the chain cannot oscillate
between states), and positive recurrent (i.e.\ if there is a non-zero
probability that we do not return to a state that we start with, the
state is transient, otherwise it is recurrent; positive recurrent
states return to their starting state within a finite time).

We can show the detailed balance condition for the Markov chain---the
acceptance probability rule of the Metropolis algorithm leads to an
equation for the transition properties of the chain---and that this
implies that the Markov chain does converge to a stationary
distribution.

Consider neighbouring points on the chain $X_t$ and $X_{t+1}$ with a
joint probability
\[ p(X_t, X_{t+1}) = p(X_t|D,I)p(X_{t+1}| X_t) \] where the second
term represents the probability of moving from $X_t$ to $X_{t+1}$, thus
\begin{align*}
  p(X_t, X_{t+1}) &= p(X_t | D, I) \min( 1, \frac{p(X_{t+1} | D,I)}{p(X_t|D,I)}) \\
&= \min(p (X_t | D,I) , p(X_{t+1}|D,I) ) \\
&= p(X_{t+1} | D,I) \min(1, \frac{p(X_t | D,I )}{p(X_{t+1}| D,I)} ) \\
&= p(X_{t+1} | D,I) p(X_t|X_{t+1} )
\end{align*}
Thus
\[ p(X_t | D,I) p(X_{t+1} | X_t) = p(X_{t+1} | D,I) p(X_t | X_{t+1}) \]
which is the detailed balance condition.

We want to show that, if the sample $X_t$ is sampled from $p(X_t)$
then $X_{t+1}$ is also from the same PDF. We can show this by
marginalising the distribution for $X_{t+1}$.

\begin{align*} 
\int p(X_t | D,I) p(X_{t+1} | X_t) \dd{X_t} &= \int p(X_{t+1} | D,I) p (X_t | X_{t+1}) \dd{X_t} \\
&= p(X_{t+1} | D,I) \int p(X_t | X_{t+1}) \dd{X_t} \\
&= p(X_{t+1} | D,I)
\end{align*}
So, the detailed balance condition gives a positive recurrent chain.

\section{Frequentist Hypothesis Testing}
\label{sec:freq-hypoth-test}

In discussion of parameter estimation we made use of an estimated PDF
to find the parameter properties. We did not, however, address the
suitability of the PDF model in the first place.

\subsection{$p$-value testing}
\label{sec:p-value-analysis}

\begin{definition}[Null Hypothesis]
  A null hypothesis is the default position; that there is no
  relationship between two measured phenomena, for example.
\end{definition}

Let us draw data from a Gaussian distribution, and determine if they
are consistent with a null hypothesis. Let us assume a variance
$\sigma^2$. We have measured data $\set{x_i}$ for $i \in [1,10]$ and
$\sum x_i = 47.8$. We assume a Gaussian distribution with $\mu_{\rm
  model} = 4$, and $\sigma^2=2$. The variance on the estimate of the
mean, $\sigma^2_{\mu}$ is $\frac{2^2}{10} = 0.4$, and the observed
mean is $4.78$.

We want to transform to a standard normal variable,
\begin{equation}
  \label{eq:5}
  Z = \qty( \frac{\bar{x}_{\rm ob} - x\bar{x}_{\rm mod}}{\sigma_{\mu}} ) \sim \mathcal{N}(\mu=0, \sigma^2=1)
\end{equation}
For this example,
\[ Z_{\rm obs} = 1.23 \] If the data is consistent with the null
hypothesis, what is the probability of finding this value or one
larger?
This probability is the $p$-value,

\begin{definition}[$p$-value]
  The $p$-value is the probability of observing a test statistic at
  least as extreme as the one which was actually observed, assuming
  the null hypothesis is true.
  \[ p(\abs{Z} \ge \abs{Z_{\rm obs}}) = 1 - \int_{-Z_{\rm obs}}^{\rm
    obs} \frac{1}{\sqrt{2 \pi}} \exp(- \frac{z^2}{2}) \dd{z} \]
\end{definition}

In this case, $p=0.218$. The smaller the $p$-value the less credible
the null hypothesis is judged to be; the cutoff significance, below
which the null hypothesis is rejected, is usually set at $p<0.05$.

$p$-values are useful, but it does not represent a judgement on
whether the null hypothesis is true, merely if it is consistent with
the data; failure to reject the null hypothesis only implies there is
insufficient evidence to reject it.

\subsection{Student's $t$ distribution}
\label{sec:stud-t-distr}

If the variance is not assumed it is possible to estimate it from the
data by forming the statistic
\[ t_{\rm obs} = \qty( \frac{\bar{x}_{\rm obs} - \bar{x}_{\rm
    mod}}{\flatfrac{\hat{\sigma}_{\mu}}{\sqrt{n}}} ) \]
where
\[ \hat{\sigma}_{\mu} = \frac{1}{(n-1)} \sum_{i=1}^n (x_i -
\bar{x}_{\rm obs})^2 \] for $n$ the number of observations, so $n-1$
represents the number of remaining degrees of freedom after
calculating the mean from the data.  $t_{\rm obs}$ does not have a
Gaussian distribution, but is described by Student's $t$-distribution.

\begin{definition}[Student's $t$-distribution]
  \begin{equation} p(t) = \frac{\Gamma \qty( \frac{\nu+1}{2} )}{\sqrt{\nu \pi}
    \Gamma \qty( \frac{\nu}{2} ) } \qty( 1 + \frac{t^2}{\nu} )^{-
    \frac{\nu+1}{2}}
\end{equation}
for $\nu = n-1$ the degrees of freedom.
\end{definition}

In general if we divide a Gaussian random variable with a scaled
random variable from a $\chi^2$-distribution the resulting statistic
follows Student's $t$-distribution, i.e.\ if $Y$ is drawn from a
Gaussian, and $Z$ from a $\chi^2$-distribution, then
\[ X = \frac{Y}{\sqrt{\flatfrac{Z}{n}}} \] with $X$ having a student's
$t$-distribution with $n-1$ degrees of freedom.  By comparing this to
the expression for $t_{\rm obs}$ we see that $\hat{\sigma}_{\mu}^2$
has a $\chi^2$-distribution.
The two types of sample variance are the biased,
\[ \sigma = \frac{1}{n} \sum_{i=1}^n x_i^2 - \qty( \frac{1}{n}
\sum_{i=1}^n x_i )^2 \]
and the unbiased
\[ \hat{\sigma}_{\mu} = \frac{1}{(n-1)} \sum_{i=1}^n (x_i -
\bar{x}_{\rm obs})^2 \] variances, which both sum the square of the
Gaussian variables which give a $\chi^2$ distribution with $n$ degrees
of freedom.

\begin{example}
  Consider radio flux density measurements from a radio galaxy over a
  period of 6100 days, made at regular time intervals.\\
  Our model assumes that the radio emission is constant in time, so
  let's test this hypothesis.

  Null hypothesis: the galaxy has an unknown but constant flux density.

  We select a goodness-of-fit $\chi^2$-statistic to compare our data
  to the model. Our data, $\set{x_i}$ have a mean $\mu = 7.98
  \milli{\rm Jy}$, and previous experience with the instrumentation
  suggests that $\sigma = 2.7$. The $\chi^2$ statistic for the data
  evaluates to $\chi^2 = 26.76$. There were a total of 15
  measurements, but one was used to compute the mean, leaving 14
  degrees of freedom.

  So, if the data is compatible with the null hypothesis, how probable
  is getting the observed value of $\chi^2$?
  \begin{align*}
    p(\abs{\chi^2} \ge \abs{\chi^2_{\rm obs}} ) &= 1 - F(\chi^2) \\
    &= 1 - \int_0^{\chi^2_{\rm obs}} \frac{1}{\Gamma \qty( \frac{\nu}{2} ) 2^{\frac{\nu}{2}}} x^{\frac{\nu}{2} - 1} \exp( - \frac{x}{2} ) \dd{x} \\
    &= 0.02
  \end{align*}
  If the flux density were constant we would only expect 2\% of
  datasets with 15 observations under the same conditions to have a
  $\chi^2$ greater than the observed one, so there is little support
  for this null hypothesis. Additionally,
  \[ \frac{\chi^2_{\rm obs} - \ev{\chi^2}}{\sigma} = \frac{26.76 -
    14}{\sqrt{2 \times 14}} = 2.4 \equiv 2.4 \sigma \]

\end{example}

\subsection{General Hypothesis Testing}
\label{sec:gener-hypoth-test}

In general we want to compare a null hypothesis, $H_0$ to another
competing hypothesis, $H_1$. To do this we construct a test statistic,
e.h. a $t$-statistic, and based on its value, we can either reject
$H_0$ in favour of $H_1$, or reject $H_1$ in favour of $H_0$. To make
the decision we need a critical region, where $H_0$ is rejected, and
an acceptance region, where $H_0$ is accepted.

In comparison to Bayesian methods, frequentist approaches to
hypothesis testing are varied, and more specific.

\begin{itemize}
\item {\bf Goodness of fit} --- $\chi^2$ test
\item {\bf Difference of means} --- Student's $t$-distribution
\item {\bf Sample cumulative distribution function} --- Kolmogorov-Smirnov test
\item {\bf Variable Correlation} --- Sample correlation coefficient
\item {\bf Discrete random variables} --- $\chi^2$ test
\item {\bf Ratio of variances} --- $F$ test.
\end{itemize}

\subsubsection{Discrete Distributions Goodness of Fit}
\label{sec:discr-distr-goodn}

Consider the case where we have a discrete data distribution, and we
want to test goodness of fit. Examples would be CCD photon counts, or
meteor observations. The $\chi^2$-statistic can be applied
here. Suppose the model follows a discrete distribution, e.g.\ the
Poisson distribution for a total of $n$ observations. It is possible
to make a histogram of the distribution over $k$ bins, with the
expected frequency of each bin denoted $e_k$. Similarly, the data for
the value in the $k$th bin is observed with a frequency of $o_k$. We
can test the null hypothesis by comparing the observed and expected
frequencies.
\[\chi^2 = \sum_{i=1}^k \frac{(o_i - e_i)^2}{e_i} \]
For a Poisson distribution $\sigma^2 = \mu$. Under the null hypothesis
this test statistic will have a $\chi^2$ distribution with $\nu=k-1-m$
degrees of freedom, with one degree lost as the sample size is
discrete, and with $m$ parameters we wish to estimate.

\subsubsection{Kolmogorov-Smirnov Test}
\label{sec:kolm-smirn-test}

The approach of section \ref{sec:discr-distr-goodn} will be
insufficient for a small number of samples, as we will be unable to
bin finely enough to constrain the PDF model. A more useful approach
is comparing the cumulative distributions of observations with a
theoretical model. This is performed using the Kolmogorov-Smirnov test
statistic.

Let $\set{x_i}$ be a set of independent and identically distributed
random samples from an unknown population. We arrange $\set{x_i}$ in
ascending order, so that the cumulative distribution is described by
\[ S_n(x) =
\begin{cases}
  0 & x < x_1 \\ \frac{i}{n} & x_i \le x \le x_{i+1}, 1 \le i \le n-1 \\
1 & x \ge x_n
\end{cases}
\]
Which is a step function with increments of $\frac{1}{n}$ at each
sample value of $x$. Let $R(x)$ be the model CDF we wish to test for
the null hypothesis. The KS test statistic is then defined as 
\begin{equation}
  \label{eq:6}
  D_n = \max( \abs{R(x) - S_n(x)} )
\end{equation}
The distribution of $D_n$ for varying $n$ is independent of
$R(x)$. The KS test is robust and non-parametric, but is more likely
to favour a false null hypothesis than other methods.

\begin{example}
  In an experiment cars are counted as they pass in front of an
  observer, during a specific time interval.\\ To attempt to determine
  the true mean a Poisson distribution is fitted to the
  observations. We choose a model mean of $\mu=11$, and calculate the
  $\chi^2$ statistic, and find a minimum at $\ev{\mu}=11$. The
  cumulative Kolmogorov-Smirnov test indicates a mean of $11$.
\end{example}

\subsubsection{Sample Correlation Coefficient}
\label{sec:sample-corr-coeff}

Recall the correlation coefficient from equation (\ref{eq:corrco}),
and noting that $\sigma_{xy}^2$ is the variance, we can expand this to
\begin{align*}
  \frac{\sigma_{xy}^2}{\sigma_x \sigma_y} &= \frac{E[(x-\mu_x)(y-\mu_y)]}{\sqrt{E[(x-\mu_x)^2] E[(y-\mu_y)^2]}} \\
&= \frac{\sum_{i=1}^n (x-\mu_x)(y-\mu_y)}{ \sqrt{\sum_{i=1}^n (x-\mu_x)^2 \sum_{i=1}^n (y-\mu_y)^2}}
\end{align*}
For the sample correlation coefficient we use
\begin{equation}
  \label{eq:7}
  \hat{\rho} = \frac{\sum_{i=1}^n (x-\hat{\mu}_x)(y-\hat{\mu}_y)}{ \sqrt{\sum_{i=1}^n (x-\hat{\mu}_x)^2 \sum_{i=1}^n (y-\hat{\mu}_y)^2}}
\end{equation}
where quantities with hats are estimated quantities from the observed
data. To test hypotheses we need the sample distribution of
$\hat{\rho}$. We consider two cases where $x$ and $y$ have a bivariate
normal PDF. If $x$ and $y$ are independent the $t$ distribution will
follow a Student's $t$ distribution,
\[ t = \frac{\hat{\rho} \sqrt{n-2}}{\sqrt{1-\hat{\rho}^2}} \] For a
large number of samples, where $x$ and $y$ have a correlation
coefficient $\rho_0$, the statistic
\[ z= \half \log( \frac{1+\hat{\rho}}{1-\hat{\rho}}) \]
will follow a Gaussian distribution with
\[ \mu_z = \half \log( \frac{1 + \rho_0}{1 - \rho_0} ) \] 
and 
\[ \sigma_z^2 = \frac{1}{n-3} \]

\section{A Frequentist Approach to Confidence Intervals}
\label{sec:freq-appr-conf}

Let $\set{x_i}$ be a set of $n=10$ samples drawn from $\nabla(\mu,
\sigma^2)$ with an unknown $\mu$ but known $\sigma=1$. The sample
mean, $\bar{x}$, can be determined with a standard deviation
$\sigma_{\mu}$,
\[ \sigma_{\mu} = \frac{\sigma}{\sqrt{10}} \approx 0.32 \]
Thus
\begin{align*}
  p(\mu-0.32 < \bar{x} < \mu+0.32) & \equiv p(\bar{x} - 0.32 < \mu <
  \bar{x} + 0.32) \\ &= 0.68
\end{align*}
Suppose we find the mean to be $\bar{x}=5.40$ from the data. We cannot then say
\[ p(5.08 < \mu < 5.72)=0.68 \], as the true mean is a fixed but
unknown quantity in frequentist analysis, so the probability that the
mean is in a given range is either 0 or 1. Instead this statement
means that when 10 samples are drawn fro $\mathcal{N}(\mu, \sigma^2)$
we expect 68\% of the samples to lie in the range. This 68\% is the
\emph{coverage}.

\section{Bayesian Model Selection}
\label{sec:bayes-model-select}

In a Bayesian framework, when we have two competing hypotheses we can
take the ratio of the posteriors for each model to form the \emph{odds
  ratio}. Given two hypotheses, $A$ and $B$,
\begin{equation}
  \label{eq:8}
  \mathcal{O}_{AB} = \frac{p(M_A|D,I)}{p(M_B | D,I)} = \frac{p(M_A|I)}{p(M_B|I)} \frac{p(D|M_A,I)}{p(D|M_B,I)}
\end{equation}
The odds ratio then gives a means to find the preferred hypothesis, as
when $\mathcal{O}_{AB}>1$ the implication is that $M_A$ is preferred,
otherwise $M_B$ is preferred. If $\mathcal{O}_{AB}=1$ there is
insufficient reason to prefer either model.

When models have numerous parameters we must marginalise over all of
the desired model parameters.

\subsection{Occam Factor}
\label{sec:occam-factor}

Again, consider two models $M_A$ and $M_B$, which both have a Gaussian
likelihood. $M_A$ has no parameters, but $M_B$ has one;
$\theta$. Clearly it will be easier to make a hypothesis like $M_B$
fit the data, as it has more degrees of freedom than $M_A$. This will
be reflected in the constant produced from marginalising the
likelihood function of $M_B$. This constant is the Occam factor, and will have the form
\begin{equation}
  \label{eq:9}
  O = \frac{b-a}{\sigma_{\theta} \sqrt{2 \pi}}
\end{equation}
in the case that $M_B$ has a Gaussian likelihood, and uniform prior
from $a$ to $b$.

\section{Choosing Priors}
\label{sec:choosing-priors}

\subsection{The Jeffreys Prior}
\label{sec:jeffreys-prior}

Uninformative priors are ones which do not significantly affect the
posterior distribution, for example, uniform priors, while vague
priors have a large spread over the parameter of choice, for example,
a Gaussian prior with a large variance. Conversely, informative priors
convey information suggesting a preference for particular parameter
values.\\
The \emph{Principle of Indifference}, which originated from
Bernoulli's \emph{Principle of Insufficient Reason}, leads us to
assign the same probability to all events for which we have no reason
to favour over each other.

Consider a parameter, $X$, which describes a continuous quantity, for
example, the location of a lighthouse along a straight
coastline. Assume that it lies somewhere within an infinitesimally
small range, $X \in (x, x+\Delta)$, so
\[ p(X=x|I) \dd{X} = \lim_{\Delta \to 0} p(x \le X \le x+\Delta | I) \]
the small change in $X$ will lead to a small change in the PDF, so
\[ p(X|I)\dd{X} \approx p(X+\Delta|I) \dd{(X+\Delta)} \]
and, since $\Delta$ is constant, $\dd{X+\Delta} = \dd{X}$, so,
\[ p(X|I) \approx \text{constant} \]

Similarly, consider the case of a scale parameter, $L$, e.g.\ the
distance to an astronomical object. From the Principle of Least
Indifference,
\[ p(L|I) \dd{L} = p(\beta L |I) \dd{(\beta L)} \] for a positive
constant $\beta$. Since 
\begin{align*}
  p(L|I) &= p(\beta L|I)\beta \\ \propto \frac{1}{L}
\end{align*}
This form of the prior on the scale parameter is a Jeffreys prior,
representing complete ignorance of the scale parameter's value. It is
the equivalent of the uniform prior for the logarithm of the scale
parameter., i.e.\
\[ p(\log(L) | I) = \tet{constant} \]

The Jeffrey's prior is an example of a much more general result.

\subsection{Fisher Information}
\label{sec:fisher-information}

Suppose our inference problem is described by a likelihood with
paramters $\vec{\theta}$, then the Jeffreys prior is an uninformative
prior defined as
\[ p(\vec{\theta}) \propto \qty[ \det(I(\vec{\theta})) ]^{\half} \]
where $I(\vec{\theta})$ is the Fisher Information,
\begin{equation}
  \label{eq:11}
  I(\vec{\theta})_{ij} = E \qty[ \pdv{\theta_i} \log(l(\vec{\theta})) \pdv{\theta_j} \log(l(\vec{\theta}))]
\end{equation}
which reduces to the Fisher Information Matrix in the case of a
Gaussian likelihood.

\subsection{Invariance}
\label{sec:invariance}

An important property of the Jeffreys prior is that it is invariant
under any reparameterisation of $\vec{\theta}$.

In order to reparameterise the prior, from $\vec{\theta} \to \vec{\phi}$ we need
\[ p(\vec{\phi}) = p(\theta) \abs{\det{J}} \]
where $J$ is a Jacobian. Let
\[ p(\vec{\theta}) \propto \qty[ \det(I(\vec{\theta}) ) ]^{\half} \] and
require the same for the reparameterised prior, $p(\vec{\phi})$.
Combining,
\[ p(\vec{\phi}) \propto \qty( \det(I(\vec{\theta}) ) )^{\half}
\det(J) = \sqrt{\det[I(\vec{\theta})JJ]} \]
Then,
\begin{align*}
I(\vec{\theta})J &=
\begin{pmatrix}
  \pdv{l}{\theta_1} \pdv{l}{\theta_1} & \cdots & \pdv{l}{\theta_1} \pdv{l}{\theta_n} \\
  \vdots & \ddots & \vdots \\
  \pdv{l}{\theta_n} \pdv{l}{\theta_1} & \cdots & \pdv{l}{\theta_n}
  \pdv{l}{\theta_n}
\end{pmatrix}
\begin{pmatrix}
  \pdv{\theta_1}{\phi_1} & \cdots & \pdv{\theta_1}{\phi_n} \\
  \vdots & \ddots & \vdots \\
  \pdv{\theta_n}{\phi_1} & \cdots & \pdv{\theta_n}{\phi_n}
\end{pmatrix}
\\ &=
\begin{pmatrix}
  \pdv{l}{\theta_1} \pdv{l}{\phi_1} & \cdots & \pdv{l}{\theta_1} \pdv{l}{\phi_n} \\
  \vdots & \ddots & \vdots \\
  \pdv{l}{\theta_n} \pdv{l}{\phi_1} & \cdots & \pdv{l}{\theta_n}
  \pdv{l}{\phi_n} \\
\end{pmatrix}\\
&:= A
\end{align*}
\begin{align*}
  p(\vec{\phi}) &\propto \sqrt{\det(AJ)} \\ &= \sqrt{\det(A) \det(J)}
  \\ &= \sqrt{\det(A^{\rm T}) \det(J)} \\ &= \sqrt{\det(A^{\rm T}J)}
\end{align*}
So,
\begin{align*}
  A^{\rm T}J &= 
\begin{pmatrix}
  \pdv{l}{\theta_1} \pdv{l}{\phi_1} & \cdots & \pdv{l}{\theta_1} \pdv{l}{\phi_n} \\
  \vdots & \ddots & \vdots \\
  \pdv{l}{\theta_n} \pdv{l}{\phi_1} & \cdots & \pdv{l}{\theta_n}
  \pdv{l}{\phi_n} \\
\end{pmatrix}
\begin{pmatrix}
  \pdv{\theta_1}{\phi_1} & \cdots & \pdv{\theta_1}{\phi_n} \\
  \vdots & \ddots & \vdots \\
  \pdv{\theta_n}{\phi_1} & \cdots & \pdv{\theta_n}{\phi_n}
\end{pmatrix}
\\ &=
\begin{pmatrix}
  \pdv{l}{\phi_1} \pdv{l}{\phi_1} & \cdots & \pdv{l}{\phi_1} \pdv{l}{\phi_n} \\
  \vdots & \ddots & \vdots \\
  \pdv{l}{\phi_n} \pdv{l}{\phi_1} & \cdots & \pdv{l}{\phi_n}
  \pdv{l}{\phi_n} \\
\end{pmatrix}\\
& := I(\vec{\phi})
\end{align*}
Thus,
\[ p(\vec{\phi}) \propto \sqrt{\det( I(\vec{\phi}) )} \] and the
Jeffreys prior is invariant under a parameter change, where a uniform
prior can go from being uninformative to informative under a
transform.

\end{document}